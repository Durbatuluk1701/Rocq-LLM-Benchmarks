\documentclass[xetex,format=acmlarge,screen=true,authorversion=true]{acmart-modified}
\usepackage{amsmath}
\usepackage{amssymb}
\usepackage{color}
%\usepackage{fontspec}
\usepackage{graphicx}
%\usepackage[bookmarks=true,bookmarksopen=true,colorlinks=true,linkcolor=blue,citecolor=blue,urlcolor=blue]{hyperref}
\usepackage{listings}
\usepackage{xcolor}
% Define the Tango colors
\definecolor{butter1}{HTML}{FCE94F}
\definecolor{butter2}{HTML}{EDD400}
\definecolor{butter3}{HTML}{C4A000}
\definecolor{orange1}{HTML}{FCAF3E}
\definecolor{orange2}{HTML}{F57900}
\definecolor{orange3}{HTML}{CE5C00}
\definecolor{chocolate1}{HTML}{E9B96E}
\definecolor{chocolate2}{HTML}{C17D11}
\definecolor{chocolate3}{HTML}{8F5902}
\definecolor{chameleon1}{HTML}{8AE234}
\definecolor{chameleon2}{HTML}{73D216}
\definecolor{chameleon3}{HTML}{4E9A06}
\definecolor{skyblue1}{HTML}{729FCF}
\definecolor{skyblue2}{HTML}{3465A4}
\definecolor{skyblue3}{HTML}{204A87}
\definecolor{plum1}{HTML}{AD7FA8}
\definecolor{plum2}{HTML}{75507B}
\definecolor{plum3}{HTML}{5C3566}
\definecolor{red1}{HTML}{EF2929}
\definecolor{red2}{HTML}{CC0000}
\definecolor{red3}{HTML}{A40000}
\definecolor{aluminium1}{HTML}{EEEEEC}
\definecolor{aluminium2}{HTML}{D3D7CF}
\definecolor{aluminium3}{HTML}{BABDB6}
\definecolor{slate1}{HTML}{888A85}
\definecolor{slate2}{HTML}{555753}
\definecolor{slate3}{HTML}{2E3436}

\definecolor{dktest}{rgb}{0.9,0.9,0.9}
\definecolor{ltblue}{rgb}{0,0.4,0.4}
\definecolor{dkblue}{rgb}{0,0.1,0.6}
\definecolor{dkgreen}{rgb}{0,0.2,0}
\definecolor{dkviolet}{rgb}{0.3,0,0.5}
\definecolor{dkred}{rgb}{0.5,0,0}
\usepackage{lstcoq}
\usepackage{adjustbox}
\usepackage{tikz}
\usepackage{tikzpagenodes}
\usepackage{enumitem}
\setitemize{itemsep=0.5em}

\newcommand{\grule}{\rule{0.8\textwidth}{.4pt}}

%% % ------------------------------------------------------------------------------
%% % Headings.

\begin{document}

\title{%
  Procrastination
}
\subtitle{%
  A proof engineering technique
}

\maketitle

\vspace{-1em}

% ------------------------------------------------------------------------------

\begin{flushright}
  {\sc Have you seen those procrastinators the monks use? \\
    Wonderful things. They can move time, store it, stretch it... \\
    quite ingenious. }

  \vspace{0.3em}

  --- Terry Pratchett, \emph{Thief of Time}
\end{flushright}

\vspace{1em}

\section*{Abstract}
We present a small Coq library for collecting side conditions and deferring
their proof.

\section{Introduction}

What are the different ways of proving an assertion of the form ``\co{exists x,
  P x}''? From the perspective of the final proof term, there is only one:
providing a witness. However, from a ``proof engineering'' perspective, Coq
enables a wider range of possible approaches thanks to its powerful feature of
``evars'' \cite[\S2.11]{coq}.

We describe existing approaches for proving such a goal, and propose a new one,
supported by a small library dubbed \emph{Procrastination}
\cite{procrastination}, which itself relies on evars. We believe it captures a
useful proof pattern, where one wishes to first continue with the proof of \co{P
  x}, in order to discover which constraints \co{x} must satisfy, before
providing a value for \co{x} and proving the side conditions.

We used this library with success in our work on verifying the asymptotic
complexity of programs \cite{gueneau-chargueraud-pottier-coq-bigO}.

\section{Proving existential quantifications in Coq}

\subsubsection*{Manually guessing a witness}
%
The most straightforward option to prove a Coq goal ``\co{exists x, P x}'' is to
guess and provide upfront a witness, using the \texttt{exists} tactic. This is
often a good way of providing external knowledge about how the proof should
continue.

\subsubsection*{Delaying the instantiation using \emph{evars}}
%
However this falls short if the only intuition is that \texttt{x} should be
``whatever makes
%% the rest of
the proof work''. A trial-and-error process can be attempted, where the user
alternatively adjusts the witness and the proof until it passes. This does not
lead to maintainable proof scripts: in the end, only a ``magic'' witness
remains; it may be large and duplicate information; it may also have to
change in the future, following updates to the rest of the proof.

To allow delaying the instantiation of \texttt{x}, Coq provides a versatile
feature: evars. An evar corresponds to a ``hole'' in the proof, and only exists
during the pre-typing phase -- it will not appear in the proof term, and only
helps elaborating it. Using the \texttt{eexists} tactic on the goal
``\co{exists x, P x}'' turns it into ``\co{P ?x}'', introducing a fresh evar
\co{?x}. Later, it can be discovered that this hole should be (definitionally)
equal to some term: the evar then gets instantiated. %% with the desired term
%
This is generally performed automatically by Coq tactics through unification.


%% There are of course restrictions on the terms that can be used to instantiate an
%% evar. For an instantiation to make sense, it has to be valid in the context in
%% which the evar was first introduced. An evar therefore carries around a context
%% that restricts its possible instantiations.

% NB: les evars sont et restent le mécanisme central qui permet de retarder
% l'instantiation d'existentielles. ce qui est présenté après sont différents
% workflows ou patterns pour les utiliser, mais ultimement sont construits par
% dessus la feature d'evars qui est très expressive, et n'étendent pas coq avec
% de nouvelles features.

\subsubsection*{``Big enough'' existentials}

This technique again falls short if continuing with the proof of \co{P ?x} does
not exhibit a term that \co{?x} must be \emph{equal} to, but e.g. \emph{lower
  bounds} on \co{?x}. Cohen \cite[\S5.1.4]{cohen-12} shows that it is a common
situation in analysis, where a proof is established by picking a ``big enough
\co{x}''. During the proof, inequalities of the form ``\co{a <= x}'', ``\co{b <=
  x}'', ... are all trivially satisfied provided that ``\co{x} is big enough''.

%% none of these would allow to instantiate \co{x} by itself,
%% but it seems undesirable to have to guess a value beforehand that satisfy all of
%% them.

Cohen formalizes this seemingly fuzzy reasoning in a small tactic library. On a
``\co{exists x, P x}'' goal, a tactic instantiates \co{x} with the iterated maximum of
a list of terms, this list being initially an evar. It is then possible to
progressively populate this list, automatically discharging subgoals of the
form ``\co{a <= x}''.
%
%The approach generalizes to any meet semi-lattice.

% Ce trick permet donc un workflow où on peut remettre à plus tard
% l'instanciation d'une existentielle ``assez grande'' pour laquelle on déduira
% des inégalités.

\section{The \emph{Procrastination} library}

We are interested in the situation where arbitrary side conditions about \co{x}
may be encountered -- e.g. inequalities of both the form ``\co{a <= x}'' and
``\co{x <= b}''. We might also be trying to infer not an arithmetic value but a
function, e.g. \co{f} satisfying ``\co{forall n, f(n) >= 1 + f(n-1)}''. In this
situation, it seems difficult to guess the shape of \co{x} or \co{f} beforehand
(e.g. as an iterated maximum).
%
% la contrainte a la forme d'une définition récursive, mais une
% evar ne peut être mentionnée dans sa définition (et il faudrait prouver que la
% def est bien fondée, hum).
%
% (RQ: (cf remarque de JH) dans ce cas précis, on pourrait peut être inférer un
% fix, ou juste une fct en récursion ouverte...)

We introduce a small Coq library named \emph{Procrastination}, which generalizes
the ideas of Cohen, and allows collecting arbitrary side conditions (to be
proved later) about zero, one or several variables. It is more general than
``big enough'' in the sense that it can deal with arbitrary side conditions, but
it does not try to solve these side conditions: this is typically done in a
second time by a dedicated procedure (e.g. for ``big enough'' this procedure
would take the running maximum of the lower-bounds collected).

In other words, the library enforces a pattern where the proof is separated in
two phases: first side conditions are collected about some variables, then they
are solved and the variables are instantiated. This is crucial in limiting the
proliferation of evars, which tend to produce brittle proofs because they can
might get incorrectly instantiated by unrelated tactics.
 \footnote{For example, we noticed that \texttt{ring\_simplify} is
   particularly careless (or sometimes crashes)
   when the goal contains evars.}

The implementation provides reusable Ltac components, with the hope that they
can help applying this technique even in more exotic scenarios. The code is
freely available online \cite{procrastination}.

\section{Example: solving recurrence (in)equations}
\label{sec:example1}

Our initial motivation for this work was analyzing the time complexity of
programs, where one has to find a solution for a set of inequations about a
program's execution time.
%
%(and typically also show that the solution admits the
%desired $O()$ bound).
%
One approach is using Cormen \emph{et al.}'s substitution
method~\cite[\S4]{cormen-en}. The idea is to guess a parameterized \emph{shape}
for the solution; substitute this shape into the goal; gather a set of
constraints the parameters must satisfy for the goal to hold; finally, show that
these constraints are indeed satisfiable.

We demonstrate this strategy using \emph{Procrastination} on cost inequations
derived from a binary search program. We ``guess'' that the solution should be a
monotonic function $f$ with shape
%
%% $
%% \lambda n. \;
%% \begin{cases}
%% a \log{n} + b & \text{if } 0 < n \\
%% c & \text{if } n = 0
%% \end{cases}
%% $
%
$\lambda n. \; \text{if} \; n = 0 \; \text{then} \; c \; \text{else} \; (a \log{n} + b)$;
%
where $a$, $b$ and $c$ are parameters, about which we wish to gradually
accumulate a set of constraints.

\vspace{0.5em}

\begin{minipage}{0.45\textwidth}
\begin{coql}
Goal exists (f: nat -> nat),
  1 <= f 0 /\
  monotonic f /\
  forall n, 0 < n -> 2 + f (n/2) <= f n.
Proof.
  /&begin defer assuming a b c&/. /*(1)*/
  \exists (fun n => if zerop n then c else a * log2 n + b).
  repeat split.
  { simpl. /*(2)*/ /&defer&/. /*(3)*/ }
  { ... (* c <= b *) /&defer&/. ... }
  { intros.
    ... (* 2 + c <= b *) /&defer&/.
    ... (* 2 <= a *) /&defer&/. }
  /&end defer&/.
  /*(4)*/ eomega.
Qed.
\end{coql}
\end{minipage}%
\hfill%
\begin{minipage}{0.5\textwidth}
\begin{adjustbox}{minipage={0.97\textwidth},fbox}
\begin{coql}
a: ?x
b: ?x0
c: ?x1
g: Group (?G a b c)
@*\grule*@
 exists f: nat -> nat,
   1 <= f 0 /\ monotonic f /\ forall n, 0 < n -> 2 + f (n/2) <= f n
\end{coql}%
\nointerlineskip%
\makebox[0pt]{%
  \begin{tikzpicture}[overlay]
    \node at (7.4, 2.7) { \footnotesize{(1)} };
  \end{tikzpicture}%
}%
\end{adjustbox}

\vspace{0.3em}

\begin{adjustbox}{minipage={0.45\textwidth},fbox}
\begin{coql}
a, b, c: nat
g: Group (?G a b c)
@*\grule*@
 1 <= c
\end{coql}
\nointerlineskip%
\begin{tikzpicture}[overlay]
  \node at (3.3, 1.55) { \footnotesize{(2)} };
\end{tikzpicture}
\end{adjustbox}%
\hfill%
\begin{adjustbox}{minipage={0.45\textwidth},fbox}
\begin{coql}
a, b, c: nat
g: Group (1 <= c /\ ?Goal0)
@*\grule*@
 Subgoal solved
\end{coql}
\nointerlineskip%
\begin{tikzpicture}[overlay]
  \node at (3.3, 1.55) { \footnotesize{(3)} };
\end{tikzpicture}
\end{adjustbox}

\vspace{0.3em}

\begin{adjustbox}{minipage={0.97\textwidth},fbox}
\vspace{-0.7em}
\begin{coql}
@*\grule*@
 exists a b c, 1 <= c /\ c <= b /\ 2 + c <= b /\ 2 <= a
\end{coql}
\nointerlineskip%
\makebox[0pt]{%
\begin{tikzpicture}[overlay]
  \node at (7.4, 0.55) { \footnotesize{(4)} };
\end{tikzpicture}%
}
\end{adjustbox}
\end{minipage}


\vspace{0.5em}

The script starts with \texttt{begin defer assuming a b c} which
introduces the parameters, as well as a ``group'' $g$, using an evar to collect
the constraints (1). Note that $a$, $b$ and $c$ are not evars themselves,
instead they appear as abstract variables. Note also that
the type of these variables
%
is inferred to be \co{nat} only further on in the proof script. After
instantiating the function with the proposed solution, the first inequality
simplifies to a condition on $c$ (2). This condition is stored (in $g$) for
later retrieval, using the \texttt{defer} tactic, which discharges the
subgoal (3). Similarly, the other two inequalities yield side conditions about
$a$, $b$ and $c$ that can be \texttt{defer}red. Finally, \texttt{end
  defer} retrieves the collected conditions, which the user must then
prove. Here, this is done by an ad hoc user-defined tactic, \co{eomega}.

Observe that the final proof script never mentions the actual witnesses: we
expect the proof to be robust in the face of minor changes in the cost
inequations, as long as the \emph{shape} of the solution remains valid.

\section{Example: simplifying recurrence (in)equations}

An other application of the approach, also coming from our work on complexity
analysis, is to simplify (in)equations about a function. The inequalities
constituting the goal of the example in Section~\ref{sec:example1} have in fact
already been simplified. In practice, analyzing the complexity of a program
yields a set of constraints $C$ and a goal $\exists f, C(f)$. $C$ has typically
been elaborated automatically from the program, and one wants to simplify the
goal into $\exists f, C'(f)$, where $C'$ is a simpler set of constraints.

In this example we show that we can use \emph{Procrastination} to collect
constraints not only about natural numbers (as in the previous example), but
also about functions. Starting from constraints (on the program execution time)
that are close to what an automated program analysis would produce, we are able
to simplify them to finally obtain the simpler constraints of the previous
example.

\vspace{1em}

\begin{minipage}{0.5\textwidth}
\begin{coql}
Goal exists (f: nat -> nat),
  forall n,
    1 + (if zerop n then 0
         else 1 + max (f (n/2)) (f (n - (n/2) - 1)))
    <= f n.
Proof.
  /&begin defer assuming f&/. /*(1)*/ \exists f.
  intro n. destruct (zerop n) as [|Hn].
  { subst n. simpl. /*(2)*/ /&defer&/. }
  { rewrite max_l; swap 1 2.
    { /*(3)*/ /&defer M: (monotonic f)&/. /*(4)*/ ... }
    { ... /*(5)*/ revert n Hn. /*(6)*/ /&defer&/. } }
  /&end defer&/.
  /*(7)*/ ...
Qed.
\end{coql}

\vspace{1.5em}

\begin{adjustbox}{minipage={0.97\textwidth},fbox}
\begin{coql}
f: ?x
g: Group (?G f)
@*\grule*@
 exists (f0: nat -> nat),
   forall n, 1 + (if zerop n then 0
           else 1 + max (f0 (n/2)) (f0 (n - (n/2) - 1)))
       <= f0 n
\end{coql}%
\nointerlineskip%
\makebox[0pt]{%
  \begin{tikzpicture}[overlay]
    \node at (7.4, 2.7) { \footnotesize{(1)} };
  \end{tikzpicture}%
}%
\end{adjustbox}

\vspace{0.3em}

\begin{adjustbox}{minipage={0.97\textwidth},fbox}
\begin{coql}
f: nat -> nat
g: Group (?G f)
@*\grule*@
 1 <= f 0
\end{coql}
\nointerlineskip%
\begin{tikzpicture}[overlay]
  \node at (7.4, 1.55) { \footnotesize{(2)} };
\end{tikzpicture}
\end{adjustbox}%

\vspace{0.3em}

\end{minipage}%
\hfill%
\begin{minipage}{0.45\textwidth}

\begin{adjustbox}{minipage={0.97\textwidth},fbox}
\begin{coql}
f: nat -> nat
g: Group (1 <= f 0 /\ ?Goal0)
n: nat
H: 0 < n
@*\grule*@
 f (n - n / 2 - 1) <= f (n / 2)
\end{coql}
\nointerlineskip%
\begin{tikzpicture}[overlay]
  \node at (6.6, 2.3) { \footnotesize{(3)} };
\end{tikzpicture}
\end{adjustbox}%

\vspace{0.3em}

\begin{adjustbox}{minipage={0.97\textwidth},fbox}
\begin{coql}
f: nat -> nat
g: Group (1 <= f 0 /\ monotonic f /\ ?Goal0)
n: nat
H: 0 < n
M: monotonic f
@*\grule*@
 f (n - n / 2 - 1) <= f (n / 2)
\end{coql}
\nointerlineskip%
\begin{tikzpicture}[overlay]
  \node at (6.6, 2.7) { \footnotesize{(4)} };
\end{tikzpicture}
\end{adjustbox}%

\vspace{0.3em}

\begin{adjustbox}{minipage={0.97\textwidth},fbox}
\begin{coql}
f: nat -> nat
g: Group (1 <= f 0 /\ monotonic f /\ ?Goal0)
n: nat
H: 0 < n
@*\grule*@
 2 + f (n / 2) <= f n
\end{coql}
\nointerlineskip%
\begin{tikzpicture}[overlay]
  \node at (6.6, 2.3) { \footnotesize{(5)} };
\end{tikzpicture}
\end{adjustbox}%

\vspace{0.3em}

\begin{adjustbox}{minipage={0.97\textwidth},fbox}
\begin{coql}
f: nat -> nat
g: Group (1 <= f 0 /\ monotonic f /\ ?Goal0)
@*\grule*@
 forall n, 0 < n -> 2 + f (n / 2) <= f n
\end{coql}
\nointerlineskip%
\begin{tikzpicture}[overlay]
  \node at (6.6, 1.55) { \footnotesize{(6)} };
\end{tikzpicture}
\end{adjustbox}%

\vspace{0.3em}

\begin{adjustbox}{minipage={0.97\textwidth},fbox}
\vspace{-0.7em}
\begin{coql}
@*\grule*@
 exists f: nat -> nat,
   1 <= f 0 /\ monotonic f /\
   forall n, 0 < n -> 2 + f (n/2) <= f n
\end{coql}
\nointerlineskip%
\begin{tikzpicture}[overlay]
  \node at (6.6, 1.30) { \footnotesize{(7)} };
\end{tikzpicture}
\end{adjustbox}%
\end{minipage}


\vspace{1em}

The first step of the proof is to start collecting constraints about \co{f},
using \co{begin defer assuming f} (1). This gives us a new identifier
\co{f}, that we can use right away as a witness for the existential
quantification at the head of the goal. Case splitting first allows extracting
the inequation for the base case (2). Some work can be done in the recursive
case (3), if we additionally assume that \co{f} is monotonic (4). In that case
the \co{max ...} can be simplified to its first argument (5). The inequation at
(5) is the simplified inequation for the recursive case. However, it is not
possible to call \co{defer} at this point: the goal depends on the local
variable \co{n}, while the set of collected constraints can only depend on
\co{f}. The solution is to generalize the goal to any positive \co{n} (6), by
moving \co{n} and \co{Hn} from the local context to the goal. Finishing the
deferring process yields the set of simplified constraints, which corresponds to
the initial goal of the previous example (7).

Here again, the point of using \emph{Procrastination} is to impose a \emph{proof
  discipline} that improves the robustness of the proof script. In this example,
we assumed that the goal has been produced by a mechanized process. We would
like the proof script to be maintainable, with respect to (non-fundamental)
changes in this mechanized process. \emph{Procrastination} is used a first time
to extract a set of simplified constraints, then a second time (as detailed in
Section~\ref{sec:example1}) to find a solution for the simplified constraints,
without relying on the exact values in these constraints. By explicitly
splitting the proof into two parts that operate at two different levels of
abstraction, we can hope that if the structure of the initial goal changes, only
the first half of the proof will need to be updated.

\section{\emph{Procrastination} and the proof context}

As illustrated previously, one can think of \emph{Procrastination} as a way of
incrementally collecting some side-conditions \co{P}, \co{Q}, \co{R} about some
variables \co{a}, \co{b}, \co{c} in order to eventually produce the goal:
\co{exists a b c, P /\\ Q /\\ R}.

There are two key aspects to this process.

\paragraph{Side-conditions are collected and gathered under a common context.}
Specifically, it is the context at the point of calling``\co{begin}
\co{defer assuming a b c}'', extended with the names \co{a},
\co{b} and \co{c}. Only side-conditions \co{P}, \co{Q} and \co{R} that make
sense in this context can be deferred using \co{defer}.

More generally, to each ``defer block'' (beginning with \co{begin
  defer} and ending with \co{end defer}) corresponds a proof
context in which the side-conditions will be gathered. This is in fact
materialized by an assumption of type ``\co{Group (...)}'' added to the context
by \co{begin defer}. This assumption stores the current collection of
side-conditions, and can be named explicitly by the user.

Requiring all side-conditions to be expressed in the same context allows the
library to collect them as a conjunction in a single final subgoal \co{exists a
  b c, P /\\ Q /\\ R}. This is crucial, since the motivation is to find values
for \co{a}, \co{b}, \co{c} by inspecting the side-conditions \co{P}, \co{Q},
\co{R} altogether (using a tactic or manually).


\paragraph{When collecting side-conditions, variables \co{a}, \co{b}, \co{c} are rigid.}
Because the end goal is to find an instantiation for \co{a}, \co{b} and \co{c},
one could think of first introducing evars for these, and then try to collect
side-conditions. However, this quickly becomes unwieldy, as the evar for
collecting side-conditions will depend itself on evars for \co{a}, \co{b} and
\co{c}. This is in general tricky to handle, the main risk being that \co{a},
\co{b} or \co{c} get accidentally instantiated in undesired ways while trying
to collect side-conditions about them.

\emph{Procrastination} imposes a two-step process, where side-conditions are
first collected with respect to ``rigid'' names for \co{a}, \co{b} and \co{c}.
This limits the number of evars involved to a single one in ``\co{Group
  (...)}'', which is only manipulated by the library tactics. After the
side-conditions have been collected, the user is free to introduce evars for
\co{a}, \co{b} and \co{c}, if needed to prove the goal \co{exists a b c, P /\\ Q
  /\\ R}.

\section{Nesting \texttt{defer} blocks}

Interestingly, it is possible to nest defer blocks in an arbitrary way.
In some sense, collecting side-conditions is done in a ``well-scoped'' manner.
For example, it is possible to open a second defer block while being in
a defer block. Then, the nested block will have a larger context than
the enclosing block. When using \co{defer} on a side-condition, the user
can control in which \co{Group} (and therefore in which context) the
side-condition gets collected.

%% \section{\emph{Procrastination} and shelving}

%% %% on peut se demander quelle est la différence avec shelve/unshelve.
%% %% le point est que shelve capture un subgoal et tout son contexte.
%% %% alors que nous on capture un ensemble de subgoals sous un contexte commun.
%% %% shelve n'aide pas si on veut instantier une variable dont les différents
%% %% subgoals dépendent: quand on est dans un subgoal on ne voit pas les autres.
%% %%
%% %% alors qu'avec procrastination on récupère l'ensemble des "contraintes" et
%% %% on peut les inspecter en parallèle afin de déduire l'instanciation.
%% %% can be contrasted w/ shelve
%%
%% This is in contrast with Coq's \emph{shelving} mechanism. Just like
%% \co{defer}, \emph{shelving} discharges a subgoal by putting it away and
%% storing it for later. Then, \emph{unshelving} retrieves the collected subgoals.
%% This may seem very similar to \emph{Procrastinate} -- but there are crucial
%% differences. Shelving a subgoal captures the whole context of the subgoal. Then,
%% unshelving restores the captured subgoals, as as many distinct subgoals.
%% \emph{Procrastinate} requires the user to discharge side-conditions under a
%% common context, but in the end gathers them all in a single subgoal (given by
%% \co{end defer}).


\section{Implementing \emph{Procrastination}: core ideas}

In essence, the implementation of \emph{Procrastination} is extremely simple. As
its core, ``\co{begin defer} \co{assuming a}'' is nothing more than
``\co{eapply defer\_1}'', where \co{defer\_1} is the following
tautological lemma:

\begin{coq}
  Lemma defer_1 : forall A (P: Prop) (Q: A -> Prop),
    (forall a, Q a -> P) ->
    (exists a, Q a) ->
    P.
\end{coq}

This lemma (and the rest of the library) comes from the combination of two main
ideas, described below.

\subsection{Collecting side-conditions}

Independently from reasoning on additional parameters \co{a}, \co{b}, \co{c}...,
the first idea is the one of incrementally collecting side-conditions while
progressing through a proof.

Introducing additional facts in the middle of a proof corresponds to the usual
\emph{Cut} rule (or \co{cut} tactic). \emph{Cut} can be stated as the following
Coq lemma ``on a goal \co{P}, we can assume some facts \co{Q} if we prove them
later'':

\begin{coq}
  Lemma cut_lemma : forall (Q P: Prop), (Q -> P) -> Q -> P.
\end{coq}

When using \co{cut} (the tactic) or when doing \co{apply cut_lemma}, the user
needs to state upfront which the additional facts \co{Q} that are being assumed
(and which proof is deferred for later) -- e.g. \co{apply (cut_lemma (foo
  /\\ bar))}.

Now, interestingly, one can also delay giving \co{Q}, by calling ``\co{eapply
  cut_lemma}''. This makes use of Coq's evars mechanism, and introduces an evar
for \co{Q}, since it hasn't been determined yet.

\begin{coq}
  Goal P.
    eapply cut_lemma.
    - intro H.
      (* H: ?Q |- P *)
    - (* |- ?Q *)
\end{coq}

In the first subgoal, we get an assumption \co{H} which type is an evar. Should
\co{P} yield a subgoal \co{P'} that we may want to prove later, \co{P'} can
simply be discharged by applying \co{H}. This will cause the evar \co{?Q} to get
instantiated with \co{P'} -- requiring \co{P'} to be proved in the second
subgoal spawned by \co{eapply cut_lemma} as a side effect.

\begin{coq}
  Goal P.
    eapply cut_lemma.
    - intro H.
      (* H: ?Q |- P *)
      ...
      (* H: ?Q |- P' *)
      now apply H.
      (* H: P' *)
    - (* |- P' *)
\end{coq}

\vspace{0.8em}

This trick can be made a bit more useful by using \co{H} to discharge not only
one subgoal, but several. This can be achieved (e.g. using a small amount of
\emph{Ltac}) by instantiating \co{H} into an iterated conjunction of already
discharged subgoals, and an evar for future ones. For instance:

\begin{coq}
  Goal P.
    eapply cut_lemma.
    - intro H. (* H: ?Q |- P *) ...
      (* H: ?Q |- P' *) now eapply (proj1 H). ...
      (* H: P' /\ ?Q |- P'' *) now eapply (proj1 (proj2 H)). ...
      (* H: P' /\ P'' /\ ?Q |- ... *)
    - (* |- P' /\ P'' /\ ?Q *)
\end{coq}

\vspace{0.3em}

This is in fact precisely how the \co{defer} tactic is implemented.

\subsection{Dealing with existentials in the goal: avoiding the proliferation of evars}

Let us now consider the more specific case where \co{P} is of the form
``\co{exists x, P x}''. In order to collect side-conditions that may talk about
\co{x}, one possibility is to first call \texttt{eexists}, introducing an evar
for \co{x}, and then use the technique described previously.

\begin{coq}
  Goal exists x, P x.
    \eexists. (* |- P ?x *)
\end{coq}

As mentioned earlier, in our experience this does not scale up very well. The
risk is that might \co{?x} get instantiated by mistake while simply trying to
collect side-conditions about it, or while progressing through the proof of
\co{P ?x}.

Therefore, the second main idea is to use variants of the \co{cut} lemma for
collecting side-conditions ``under existential quantifications''. For example,
we can easily prove the following lemma:

\begin{coq}
  Lemma cut_ex1_lemma: forall A (P Q : A -> Prop),
    (forall x, Q x -> P x) ->
    (exists x, Q x) ->
    exists x, P x.
\end{coq}

By doing \co{eapply cut_ex1_lemma}, we can collect side-conditions about \co{x}
in \co{?Q x}, without introducing an evar for \co{x}. All in all, all this does
is massaging the proof so that it is setup more conveniently.

Finally, the lemma above can be generalized to support an arbitrary number of
quantified variables (including 0, in which case this corresponds to the
previous subsection). This is in essence how \co{begin defer} is
implemented.

%%\subsection{}
%%% 3) reusable ltac components

% because of pandoc
\def\tightlist{}
\input{tactics_reference}

% ------------------------------------------------------------------------------
% Bibliography.

\bibliographystyle{splncs03} % splncs (older) or splncs03 (more recent)
\hbadness=10000
\bibliography{local}

\end{document}
