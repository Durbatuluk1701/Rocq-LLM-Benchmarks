% • What is the challenge of your study problem? 
% • What is the existing method for your study problem (papers, online search results, less than 5)
%  • What is their strength and weakness
%  • How is your work similar or different from them?

\section{Related Work}
\label{sec:related-work}

\paragraph{Evaluating LLM-generated code in Coq}

The core question of this work aims to evaluate
the ability of LLMs to produce correct Coq proofs
despite input obfuscation.

Proof Automation with Large Language Models\cite{proofautomationwithllms}



\paragraph
{Creating datasets}
Learning to Prove Theorems via
Interacting with Proof Assistants


\paragraph
{Methodology in Coq against other languages}

Verifying the correctness of Coq proofs
compared against LLM-generated code for other
languages is a relatively closed problem
given the well-defined proof goal and the
strictness of the Coq theorem prover.
CodeJudge (Tong and Zhang, 2024) explores
alternative approaches of verification
employed to 

In our target domain, Coq proofs, 
we are interested in the formal correctness of our LLM output.
For Proof scripts in a theorem prover,
the correctness of the generated code is proved true
simply by executing the proof in the theorem prover.
If code type-checks without an error it is guaranteed to be correct.
In typical LLM code generation,
there are a few criteria for judging the quality of LLM generated code.  
A common metric for program correctness is behavioral correctness,
which can be accomplished by comparing the outputs of generated code
and a known correct program on the problem specification such as through unit testing.
Text similarity between ground truth code and the LLM output can be used
in cases where an LLM may produce code that is conceptually close
to correct code but features syntax errors that may render an
otherwise accurate solution unable to be compiled.
CodeJudge (Tong and Zhang, 2024) is a novel technique
used to evaluate LLM-generated code using another LLM.
The judging LLM employs "slow thinking" to arrive at a
thorough evaluation of the generated code,
which unlike unit tests or code similarity can consider
more nuanced notions of correctness and can have its
criteria adjusted.
We chose to use proof correctness as evaluation criteria
based upon the objectives of this research.
If the use case for LLM proof generation is to
collaborate with a human programmer,
more flexible evaluation criteria, as those above, may be more relevant. 

CodeJudge: Evaluating Code Generation 
with Large Language Models
