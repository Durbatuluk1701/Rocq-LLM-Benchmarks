% This is samplepaper.tex, a sample chapter demonstrating the
% LLNCS macro package for Springer Computer Science proceedings;
% Version 2.20 of 2017/10/04
%
\documentclass[runningheads]{llncs}
%
\usepackage{graphicx}
% Used for displaying a sample figure. If possible, figure files should
% be included in EPS format.
%\usepackage{tikz}
%\usetikzlibrary{arrows}
\usepackage{verbatim}
\usepackage{algorithm}
\usepackage[noend]{algpseudocode}
\usepackage{amssymb}
\usepackage{amsfonts}
\usepackage{amsmath}
\let\proof\relax\let\endproof\relax
\usepackage{amsthm}
% \usepackage{graphicx}
%\usepackage[all]{xy}
\usepackage{array}
\usepackage{enumitem}
%\usepackage{cite}
\usepackage[numbers,sectionbib]{natbib}
\usepackage{wrapfig}
\theoremstyle{definition}
\renewcommand{\qedsymbol}{\hfill\ensuremath{\blacksquare}}
%\newtheorem{definition}{Definition}[section]
% If you use the hyperref package, please uncomment the following line
% to display URLs in blue roman font according to Springer's eBook style:
% \renewcommand\UrlFont{\color{blue}\rmfamily}
\usepackage[breaklinks=true]{hyperref}
\usepackage{breakcites}
\renewcommand\UrlFont{\color{blue}\rmfamily}

\begin{document}
%
\title{Adaptive Theorem Proving Benchmarks for Robust LLM Evaluation}
%
%\titlerunning{Abbreviated paper title}
% If the paper title is too long for the running head, you can set
% an abbreviated paper title here
%
\author{Will Thomas \and Logan Schmalz \and Bryan Richlinski \and Jack Reynolds}

\institute{Institute for Information Sciences \\ The
  University of Kansas \\ Lawrence, KS 66045 \\
  \email{\{30wthomas, loganschmalz, b748r023, jackreynolds\}@ku.edu}}
%
\maketitle              % typeset the header of the contribution
%
We propose a core semantics for Dependent Haskell, an extension of Haskell
with full-spectrum dependent types. Our semantics consists of two related
languages. The first is a Curry-style dependently-typed language with
nontermination, irrelevant arguments, and equality abstraction. The second,
inspired by the Glasgow Haskell Compiler's core language FC, is its
explicitly-typed analogue, suitable for implementation in GHC. All of our
results---chiefly, type safety, along with theorems that relate these two
languages---have been formalized using the Coq proof assistant. Because our
work is backwards compatible with Haskell, our type safety proof holds in the
presence of nonterminating computation. However, unlike other full-spectrum
dependently-typed languages, such as Coq, Agda or Idris, because of this
nontermination, Haskell's term language does not correspond to a consistent
logic.

%
%
%
\section{Introduction}
\label{sec:introduction}

\subsection{Background and Motivation}
\label{sec:intro_background}

Large Language Models (LLMs) have demonstrated remarkable capabilities across a diverse array of natural language processing tasks and are increasingly being explored for their potential in specialized domains requiring complex reasoning, such as software development, mathematics, and formal verification. Within the realm of computer science, formal methods provide rigorous, mathematically-grounded techniques for specifying, developing, and verifying software and hardware systems. Theorem provers, such as the Coq proof assistant, are pivotal tools in formal methods, enabling users to construct machine-checked proofs of correctness. The prospect of leveraging LLMs to assist or even automate aspects of the formal proof development process in Coq could significantly lower the barrier to entry for formal methods and enhance productivity for experienced users.

However, the effective integration of LLMs into formal reasoning workflows necessitates a clear understanding of their actual deductive capabilities, as distinct from their ability to recall or replicate patterns seen during training. As LLMs are trained on vast swathes of publicly available text and code, including potentially many formal proofs and tutorials, their performance on standard benchmarks may not accurately reflect true generalization or reasoning prowess.

\subsection{Problem Statement}
\label{sec:intro_problem}

A critical issue in evaluating LLMs for theorem proving is the risk of benchmark contamination or ``memorization.'' Many established benchmarks for formal methods consist of static sets of problems that may have been part of the LLMs' training data. Consequently, an LLM might successfully ``solve'' a benchmark problem not by deducing the solution, but by recalling it, or parts of it, from its training set. This phenomenon makes it difficult to:
\begin{enumerate}
    \item Reliably compare the true reasoning abilities of different LLMs.
    \item Ascertain whether LLMs genuinely ``understand'' the underlying semantics of formal statements and proof strategies.
    \item Gauge their robustness when faced with novel-looking but semantically equivalent problems.
\end{enumerate}

There is, therefore, a pressing need for evaluation methodologies that can mitigate the effects of data leakage and assess an LLM's ability to generalize its reasoning capabilities beyond an exact replication of previously encountered examples.

\subsection{Proposed Solution and Contributions}
\label{sec:intro_solution}

This project addresses the aforementioned challenges by developing and evaluating a dynamic benchmark generation strategy for Coq theorem proving tasks. The core idea is to create new test cases by applying programmatic transformations to existing, well-understood theorems, specifically those from the widely recognized \emph{Logical Foundations} series by \citet{PierceSF:LF2024}. These transformations are designed to alter the syntactic appearance of the theorems while preserving their underlying logical semantics and provability. In this study, the primary transformation implemented is the systematic renaming of identifiers (such as variables and local definitions) within the Coq theorem statements and their contexts.

The experimental methodology involved the following steps:
\begin{enumerate}
    \item \textbf{Data Collection:} Coq files were sourced from the \emph{Logical Foundations} dataset.
    \item \textbf{Theorem Mutation:} A transformation process was applied to these files to automatically rename identifiers, creating a parallel set of mutated (perturbed) files.
    \item \textbf{Task Generation:} From both the original and mutated files, individual theorem statements were extracted.
    \item \textbf{LLM Evaluation:} Two contemporary LLMs, \texttt{llama3.1} and \texttt{phi4}, were prompted to generate proofs for each extracted theorem (both original and its renamed counterpart).
    \item \textbf{Verification:} The LLM-generated outputs were first assessed for structural usability (i.e., whether they formed a coherent Coq proof script). Usable scripts were then compiled using the Coq compiler (\texttt{coqc}) to rigorously verify their correctness.
\end{enumerate}

By comparing the LLMs' success rates and the nature of their outputs on original versus renamed theorems, this study aims to shed light on their sensitivity to superficial syntactic variations. A significant drop in performance on renamed theorems would suggest a reliance on memorized surface patterns, whereas comparable performance would indicate a more robust, semantic understanding.

The \textbf{primary contributions} of this work are:
\begin{itemize}
    \item An implemented methodology and tool for generating perturbed Coq theorem proving tasks based on identifier renaming, aimed at reducing benchmark leakage in LLM evaluations.
    \item An empirical evaluation of two modern LLMs (\texttt{llama3.1} and \texttt{phi4}) on this dynamic benchmark, providing initial data on their performance and consistency when faced with original and syntactically varied theorems.
    \item Insights into the current capabilities and limitations of LLMs in the context of formal theorem proving, specifically concerning their ability to generalize beyond surface-level syntax.
\end{itemize}

This research represents a step towards developing more authentic and reliable assessments of LLM reasoning abilities in the demanding domain of formal methods.

\subsection{Report Outline}
\label{sec:intro_outline}

The remainder of this report is structured as follows:
\begin{itemize}
    \item \textbf{Chapter 2 (Background and Related Work):} Discusses relevant literature on LLMs in code and formal reasoning, existing Coq benchmarks, program transformation techniques, and challenges in LLM evaluation.
    \item \textbf{Chapter 3 (Dataset Collection):} Details the data corpus, its significance in the Coq community, and the process of extracting theorems from the \emph{Logical Foundations} dataset.
    \item \textbf{Chapter 4 (Methodology):} Explains the design of the theorem transformation engine focusing on identifier renaming, and the overall benchmark generation process, as well as the LLM prompting strategy.
    \item \textbf{Chapter 5 (Experimental Setup):} Describes the LLMs tested, the precise task definition, the evaluation metrics employed (usability and Coq compilation), and the experimental environment.
    \item \textbf{Chapter 6 (Results):} Presents the quantitative and qualitative findings from the experiments, comparing LLM performance on original versus perturbed theorems.
    \item \textbf{Chapter 7 (Conclusion and Future Work):} Summarizes the key contributions and suggests avenues for future research in this area.
\end{itemize}
% • What is the challenge of your study problem? 
% • What is the existing method for your study problem (papers, online search results, less than 5)
%  • What is their strength and weakness
%  • How is your work similar or different from them?

\section{Related Work}
\label{sec:related-work}


\paragraph{Creating datasets}
To properly evaluate LLM performance on
Coq code generation requires a corpus of well-structured
proofs of theorems. For some language processing tasks
there are large bodies of existing works that can be leveraged for this purpose, 
with proper consideration for authorized reuse.
For popular programming
languages such as Python, C, and JavaScript this is easily achieved.
As theorem provers are relevant to a much smaller audience, the pool of quality examples in Coq is
comparatively small.
\citet{learningtoprove} compose CoqGym,
a bespoke dataset and training environment for deep learning models
composed of 70,000 human-written proofs from over 100 projects.
This work addresses the issue of scale required for LLM training,
but any LLMs produced
after the publication of this work
may have been exposed to it and cannot be judged blindly.
\textbf{Our dataset intends to address this challenge through mutating variables included in existing datasets.}

\paragraph{Evaluating LLM-generated code in Coq}

\citep{proofautomationwithllms} introduces PALM,
a novel approach for generating Coq proofs with LLMs.
PALM can utilize an existing LLM, such as GPT-4 or Llama-3,
as a baseline for proof synthesis and supplies repair tools
to refine a model's answer.
Contrasted with existing approaches that are able
to solve 15\%-20\% of theorems presented to them,
PALM achieves a proof success rate of 32\%-43\% 
depending on the base model used.
Without PALM, none of the base models can
complete more than 7\% of the test set.
PALM uses the previously mentioned CoqGym test set
for performance evaluation, 
a set that was published 5 years prior. 
In their paper, \citeauthor{proofautomationwithllms} describe the steps 
they took to avoid testing with data 
that the underlying LLMs had already seen.  
While the authors removed biased test data they were aware of, 
they used publicly available models as a starting point, 
meaning there is no guarantee that the baseline models 
they used have not been tainted with undisclosed training data. 
The probability of a model being tainted 
with the training data in the CoqGym set increases with time, 
meaning that either new datasets for Coq code will be required, 
or a means of mutating existing datasets will be required; 
the latter being the solution we pursue.


%Learning to Prove Theorems via
%Interacting with Proof Assistants

\paragraph{Methodology in Coq against other languages}
In our target domain, Coq proofs,
code can be shown to be correct by simply type-checking in the theorem prover.
LLM code generation for other languages
has been judged by several other criteria for correctness,
including behavioral correctness and code similarity.
\citet{CodeJudge} present CodeJudge, a novel technique
used to evaluate LLM-generated code using another LLM.
The judging LLM employs "slow thinking" to arrive at a
thorough evaluation of the generated code,
which unlike unit tests or code similarity can consider
more nuanced notions of correctness and can have its
criteria adjusted.
We chose to use proof correctness as evaluation criteria
based upon the objectives of this research.
If the use case for LLM proof generation is to
collaborate with a human programmer,
more flexible evaluation criteria, as those above, may be more relevant. 


% • Describe your dataset, how it is collected, feature list, data statistics (e.g., feature/target
% distribution, male/female, number of subjects)
%  • What is the data preprocessing you did (normalization, missing value, feature selection, sampling
% …)?
% • What are the inputs and output?
% • What is the data size (training/validation/testing)?

\section{Dataset Collection and Preprocessing}
\label{sec:dataset_collection_preprocessing}

The foundation of our dynamic benchmark suite is a carefully selected corpus of Coq theorems. This section details the source of these theorems, its significance within the Coq community, the programmatic methodology employed for extracting theorem statements and their associated contexts, and an overview of the resulting dataset's characteristics.

\subsection{The \emph{Logical Foundations} Data Corpus}
\label{sec:data_corpus}

For this study, the primary source of Coq theorems is the \emph{Logical Foundations} volume from the Software Foundations series by \citet{PierceSF:LF2024}. This series is a widely recognized and highly regarded resource in the formal methods and programming languages communities.

The significance of \emph{Logical Foundations} as a data source includes:
\begin{itemize}
    \item \textbf{Pedagogical Value:} It serves as a popular introductory textbook for learning the Coq proof assistant and fundamental concepts in logic, functional programming, and programming language theory (e.g., operational semantics, type systems). Its theorems are often clearly stated and accompanied by detailed, didactic proofs.
    \item \textbf{Structured Content:} The Coq files (\texttt{.v} files) associated with the book are well-structured and publicly available, making them amenable to automated processing.
    \item \textbf{Fundamental Concepts:} The theorems cover a range of foundational topics, providing a solid baseline for evaluating an LLM's reasoning capabilities on core logical constructs, inductive types, and recursive definitions commonly encountered in formal verification.
    \item \textbf{Community Relevance:} Due to its widespread adoption in academic courses and for self-study, it is plausible that LLMs may have encountered this material during their training. This makes it an interesting test case for our hypothesis that superficial transformations can effectively challenge memorization and test deeper understanding.
    \item \textbf{Manageable Scope:} The size and complexity of \emph{Logical Foundations} are substantial enough to provide a diverse set of theorems while remaining manageable for systematic extraction and experimentation within the scope of this project.
\end{itemize}
The choice of this corpus allows for the evaluation of LLMs on theorems that are both fundamental and representative of common Coq usage patterns, particularly in educational and foundational Programming Language (PL) research contexts.

\subsection{Theorem Extraction Methodology}
\label{sec:theorem_extraction}

To create our initial dataset of theorems from the \emph{Logical Foundations} \texttt{.v} files, we developed a custom Python script. The primary goal of this script was to parse each Coq source file and extract all formal statements defined using Coq's \texttt{Theorem}, \texttt{Lemma}, or \texttt{Fact} keywords, along with the necessary contextual information required to understand and prove them.

The extraction process, implemented in the \texttt{collect\_theorems} function operates as follows:
\begin{enumerate}
    \item \textbf{Input Processing:} The script takes a Coq source file (\texttt{.v} file) as input and reads its content into a string. The text is split into lines to facilitate iterative processing.
    \item \textbf{Theorem Identification:} It iterates through the lines of the file, using regular expressions to detect the declaration of theorems, lemmas, or facts. The specific pattern employed for this initial detection is \texttt{\textbackslash s*(Theorem|Lemma|Fact)\textbackslash s+([A-Za-z][A-Za-z0-9\_']*)}. This pattern captures the declaration type (e.g., ``Theorem'') and its assigned name (e.g., ``plus\_comm'').
    \item \textbf{Statement Accumulation:} Upon identifying a theorem declaration, the script accumulates all subsequent lines belonging to the statement itself. This accumulation continues until a line matching the pattern \texttt{\textbackslash s*Proof\textbackslash .} is encountered, which typically signifies the beginning of the proof script and thus the end of the theorem statement. The collected lines are concatenated to form the complete theorem statement. A trailing period is appended if not already present, ensuring syntactic completeness for standalone presentation.
    \item \textbf{Context Preservation:} For each extracted theorem, the script also captures the entire content of the Coq file that precedes the theorem's declaration line. This ``context'' is crucial, as it includes all necessary \texttt{Require}, \texttt{Import}, \texttt{Inductive} type definitions, \texttt{Fixpoint} definitions, \texttt{Notation} declarations, and prior lemmas or definitions that the current theorem might depend upon. Without this context, a theorem statement would often be unprovable or even syntactically invalid.
    \item \textbf{Structured Output:} The script compiles a list of dictionaries, where each dictionary represents an extracted theorem and contains the following key-value pairs:
    \begin{itemize}
        \item \texttt{"type"}: The type of the declaration (e.g., ``Theorem'', ``Lemma'', ``Fact'').
        \item \texttt{"name"}: The declared name of the theorem.
        \item \texttt{"statement"}: The full text of the theorem statement, up to (but not including) the \texttt{Proof.} line.
        \item \texttt{"context"}: The complete Coq code from the beginning of the file up to the line immediately preceding the theorem's declaration.
    \end{itemize}
    This structured data, typically serialized as a JSON object, then serves as the input for the subsequent transformation and LLM evaluation phases. This constitutes the primary data preprocessing step, transforming raw \texttt{.v} files into a structured dataset of theorem-context pairs.
\end{enumerate}

This regex-based approach was chosen for its simplicity and effectiveness in parsing the relatively well-structured Coq files found in \emph{Logical Foundations}. While a full Coq Abstract Syntax Tree (AST) parser could offer more robustness, the current method proved adequate for accurately extracting the required components for this project's scope.

\subsection{Data Statistics and Characteristics}
\label{sec:data_statistics}

The theorem extraction process was applied to all available Coq source files within the selected version of the \emph{Logical Foundations} dataset. This resulted in a comprehensive collection of formal statements suitable for our benchmarking task.

The key statistics of the collected base dataset are as follows:
\begin{itemize}
    \item \textbf{Total \texttt{.v} Files Processed:} 16
    \item \textbf{Total Statements Extracted:} 508
    \item \textbf{Distribution by Type:}
    \begin{itemize}
        \item \texttt{Theorem}: 389
        \item \texttt{Lemma}: 115
        \item \texttt{Fact}: 4
    \end{itemize}
    \item \textbf{Statement Length Characteristics:}
    \begin{itemize}
        \item Minimum lines: 1
        \item Maximum lines: 55
        \item Average lines: 2.64
    \end{itemize}
    \item \textbf{Context Length Characteristics:}
    \begin{itemize}
        \item Minimum lines: 24
        \item Maximum lines: 2729
        \item Average lines: 887.41
    \end{itemize}
    \item \textbf{Input/Output Definition:} For the purpose of this study, each extracted theorem forms a data point. The \textbf{input} for our subsequent LLM evaluation consists of the combined \texttt{"context"} and \texttt{"statement"}. The expected \textbf{output} from the LLM is a Coq proof script for the given statement.
    \item \textbf{Dataset Size for Evaluation:} A subset of all the above theorems was utilized for our evaluation. This subset was 424 theorems that were amiable to LLM intervention. In particular, our proof repair mechanism was unable to reconcile proof's nested within a module, \textbf{as name shadowing caused major issues}. For this reason, only the 424 extracted statements constitute our evaluation benchmark. No distinct training, validation, or testing splits are created from this dataset by our methodology, as the objective is to evaluate pre-trained LLMs on these specifically constructed problems.
\end{itemize}
This initial dataset, characterized by these statistics, provides a rich and varied set of formal reasoning problems. The subsequent transformation process builds upon this base to generate the dynamic benchmarks used for LLM evaluation.
%  • Your learning algorithms
%  • Describe the model, its mechanism, how it works in your scenario
%  • Learning objective in formal language and math equations (refers to textbook and lecture notes)
%  • What is the strength and weakness of your model?

\section{Methodology}
\label{sec:methodology}
% • 5. Experiments: 
% • What is the setting of your experiments and how you find it (settings for your model like learning 
% rate, iteration, batch size, cross-validation)
%  • Describe you training procedure
%  • Analyze if you had overfitting problem and how to handle it
%  • What are your baselines and why? 
% • List and explain your evaluation metrics (accuracy, precision, AUC…)
%  • Show results in tables and visualization
%  • Show qualitative insights from of your results
%  • Discuss your results and the future work (optional)

\section{Experiment}
\label{sec:experiment}

Our study explores the capabilities of large language models (LLMs) in interacting with formal proof code, specifically Coq files from the \textit{Logical Foundations} textbook. We used two open-source LLMs---\textbf{LLaMA 3.1} \cite{grattafiori2024llama3herdmodels} and \textbf{Phi-4} \cite{abdin2025phi4reasoningtechnicalreport}---run locally via \href{https://ollama.com}{Ollama} on an HPC cluster equipped with two Nvidia A100 GPUs. No fine-tuning was applied.

Both LLMs were queried in a \textbf{zero-shot} setting with no chat history,
where the model was given all code context from the start of the file up to a target proof obligation.
The prompt consisted of the source code up to a theorem statement with a missing proof. 
The model's task was to generate a proof that would satisfy the Coq compiler.
We wrapped each prompt in a fixed natural language template that instructed the model to generate a Coq proof body:
\begin{lstlisting}[basicstyle=\ttfamily\footnotesize,columns=fullflexible,frame=single,keepspaces=true]
Prove the following theorem {name} in the Coq proof language.
The context for the proof is as follows:
```coq
{context}
```
The statement of the theorem to be proved is as follows:
```coq
{statement}
```
Supply only the complete proof body in the Coq proof language following the template:
```coq
Proof.
    <proof body here>
Qed.
```
\end{lstlisting}

We also conducted preliminary experiments with other models available through Ollama,
including \textbf{Gemma}, \textbf{Qwen3}, \textbf{DeepSeek R1 32B}, \textbf{Mistral 7B}, and \textbf{Mixtral 8x7B}.
However, in early testing, these models did not produce valid or useful output at a high enough rate to justify running them across the full dataset.
As a result, we focused our primary experiments on \textbf{LLaMA 3.1} and \textbf{Phi-4}.

We evaluated model performance on 424 Coq proof obligations drawn from the \textit{Logical Foundations} textbook.
For each proof, we tested the original file with human-readable identifiers and the renamed file where identifiers were mutated.

\noindent\textit{Procedure:}
Each model output was inserted into the proof location and compiled with Coq. If compilation succeeded, the attempt was marked as successful.

\noindent\textit{Limitations:}
In rare instances, the LLM ignored instructions about formatting the output and we were unable to extract a proof body.
We believe this could fixed with more advanced extraction strategies or a different prompt.
We also had to run on smaller models due to lack of sufficient GPU hardware.
% • 5. Experiments: 
% • What is the setting of your experiments and how you find it (settings for your model like learning 
% rate, iteration, batch size, cross-validation)
%  • Describe you training procedure
%  • Analyze if you had overfitting problem and how to handle it
%  • What are your baselines and why? 
% • List and explain your evaluation metrics (accuracy, precision, AUC…)
%  • Show results in tables and visualization
%  • Show qualitative insights from of your results
%  • Discuss your results and the future work (optional)

\section{Experiment}
\label{sec:experiment}

Our study explores the capabilities of large language models (LLMs) in interacting with formal proof code, specifically Coq files from the \textit{Logical Foundations} textbook. We used two open-source LLMs---\textbf{LLaMA 3.1} \cite{grattafiori2024llama3herdmodels} and \textbf{Phi-4} \cite{abdin2025phi4reasoningtechnicalreport}---run locally via \href{https://ollama.com}{Ollama} on an HPC cluster equipped with two Nvidia A100 GPUs. No fine-tuning was applied.

Both LLMs were queried in a \textbf{zero-shot} setting with no chat history,
where the model was given all code context from the start of the file up to a target proof obligation.
The prompt consisted of the source code up to a theorem statement with a missing proof. 
The model's task was to generate a proof that would satisfy the Coq compiler.
We wrapped each prompt in a fixed natural language template that instructed the model to generate a Coq proof body:
\begin{lstlisting}[basicstyle=\ttfamily\footnotesize,columns=fullflexible,frame=single,keepspaces=true]
Prove the following theorem {name} in the Coq proof language.
The context for the proof is as follows:
```coq
{context}
```
The statement of the theorem to be proved is as follows:
```coq
{statement}
```
Supply only the complete proof body in the Coq proof language following the template:
```coq
Proof.
    <proof body here>
Qed.
```
\end{lstlisting}

We also conducted preliminary experiments with other models available through Ollama,
including \textbf{Gemma}, \textbf{Qwen3}, \textbf{DeepSeek R1 32B}, \textbf{Mistral 7B}, and \textbf{Mixtral 8x7B}.
However, in early testing, these models did not produce valid or useful output at a high enough rate to justify running them across the full dataset.
As a result, we focused our primary experiments on \textbf{LLaMA 3.1} and \textbf{Phi-4}.

We evaluated model performance on 424 Coq proof obligations drawn from the \textit{Logical Foundations} textbook.
For each proof, we tested the original file with human-readable identifiers and the renamed file where identifiers were mutated.

\noindent\textit{Procedure:}
Each model output was inserted into the proof location and compiled with Coq. If compilation succeeded, the attempt was marked as successful.

\noindent\textit{Limitations:}
In rare instances, the LLM ignored instructions about formatting the output and we were unable to extract a proof body.
We believe this could fixed with more advanced extraction strategies or a different prompt.
We also had to run on smaller models due to lack of sufficient GPU hardware.
%  • 6. Conclusion: what have you done, what did you find (short)

\section{Conclusion}
\label{sec:conclusion}
This report detailed the development and application of a dynamic benchmark suite designed to more authentically evaluate the theorem-proving capabilities of Large Language Models (LLMs) in the Coq proof assistant, specifically addressing the challenge of benchmark memorization. 
By systematically renaming identifiers in theorems sourced from the \emph{Logical Foundations} dataset, we generated semantically equivalent but syntactically novel challenges for two contemporary LLMs, \texttt{llama3.1} and \texttt{phi4}. Further, our approach can be re-used to generate theoretically infinite sets of syntactically diverse theorems, which can be used to evaluate LLMs in a more robust manner.

The core findings revealed that \textbf{models exhibit a discernible (but minor) decrease in their success rates when attempting to prove these mutated theorems compared to their original counterparts}. While \texttt{phi4} demonstrated higher overall performance and slightly greater robustness to these syntactic perturbations than \texttt{llama3.1}, the observed performance degradation across both models suggests a sensitivity to identifier names, implying a reliance that extends beyond pure logical structure to include learned surface-level patterns.

Overall, the results indicate that LLMs are not yet fully capable of generalizing their reasoning abilities in the context of formal theorem proving, as evidenced by their performance on the original versus mutated theorems. The models struggled to produce compilable proofs even for original theorems, highlighting the challenges inherent in this domain. The low rate of pre-compilation failures for both models suggests that they were able to generate syntactically valid outputs more consistently when faced with renamed identifiers, even if those outputs were not logically correct.


%
% 
% ---- Bibliography ----
%
% BibTeX users should specify bibliography style 'splncs04'.
% References will then be sorted and formatted in the correct style.
%
%\bibliographystyle{splncs04}
\bibliographystyle{splncsnat}
%\bibliography{sldg}
\bibliography{bib/sldg}
%
\end{document}