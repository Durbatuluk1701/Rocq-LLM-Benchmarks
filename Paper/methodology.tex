%  • Your learning algorithms
%  • Describe the model, its mechanism, how it works in your scenario
%  • Learning objective in formal language and math equations (refers to textbook and lecture notes)
%  • What is the strength and weakness of your model?

\section{Methodology}
\label{sec:methodology}
\subsection{Learning Algorithms}

Our study explores the capabilities of large language models (LLMs) in interacting with formal proof code, specifically Coq files from the \textit{Logical Foundations} textbook. We used two open-source LLMs---\textbf{LLaMA 3.1} and \textbf{Phi-4}---run locally via \href{https://ollama.com}{Ollama} on an HPC cluster equipped with an Nvidia A100 GPU. No fine-tuning was applied; all models were used in their original zero-shot inference configuration.

\subsection{Model Description and Setup}

Both LLMs were queried in a \textbf{zero-shot} setting, where the model was given all code context from the start of the file up to a target proof obligation. The prompt consisted of the modified source code and a specific location where a proof (e.g., for a \texttt{Theorem}, \texttt{Lemma}, or \texttt{Fact}) was missing. The model’s task was to generate a proof term that would satisfy the Coq compiler.

We also conducted preliminary experiments with other models available through Ollama, including \textbf{Gemma}, \textbf{Qwen3}, \textbf{DeepSeek R1 32B}, \textbf{Mistral 7B}, and \textbf{Mixtral 8x7B}. However, in early testing, these models did not produce valid or useful output at a high enough rate to justify running them across the full dataset. As a result, we focused our primary experiments on LLaMA 3.1 and Phi-4.

To isolate the role of human-readable identifiers (such as \texttt{Theorem}, \texttt{Lemma}, \texttt{Inductive}, etc.), we applied a preprocessing pipeline:

\begin{itemize}
  \item \textbf{Identifier Detection and Renaming:} We located all top-level identifiers associated with theorems, lemmas, facts, and inductive constructors, then renamed them to opaque, non-descriptive tokens.
  \item \textbf{Comment Removal:} All comments were stripped from the files to prevent LLMs from accessing potentially informative natural language.
  \item \textbf{Module Filtering:} Entire modules were excluded from the prompt context to avoid issues with internal naming scopes and to simplify parsing.
\end{itemize}

By comparing LLM performance before and after identifier renaming, we evaluated whether the models exploit these names as semantic cues.

\subsection{Learning Objective}

Our goal was not to train the models, but to evaluate their robustness under controlled perturbations in source code. Although we did not formulate a formal learning objective with a loss function, our evaluation criterion can be thought of as a binary classification over model outputs:

\[
\text{Success}(x) =
\begin{cases}
1, & \text{if Coq compiler accepts } x \\
0, & \text{otherwise}
\end{cases}
\]

where \( x \) is the full Coq file with the model-generated proof inserted. We compared the acceptance rates across versions with original vs. renamed identifiers.

Letting \( D = \{x_i\} \) be a dataset of proof goals and \( f_\theta(x_i) \) the LLM's generated proof for input \( x_i \), the empirical success rate is:

\[
\text{Accuracy} = \frac{1}{|D|} \sum_{i=1}^{|D|} \mathbb{1}\{ \text{Coq}(f_\theta(x_i)) = \text{valid} \}
\]

\subsection{Strengths and Weaknesses}
At this stage of the project, a full comparative analysis is still ongoing. However, preliminary observations suggest that both LLaMA 3.1 and Phi-4 can occasionally generate valid Coq proofs under zero-shot conditions, even when human-readable identifiers are obfuscated. Future work will involve:

\begin{itemize}
  \item Assessing proof correctness beyond compiler acceptance (e.g., comparing proof structures or logical strategies).
  \item Introducing prompt engineering strategies like few-shot examples and ``proof context'' augmentation.
  \item Evaluating sensitivity to varying identifier schemes and levels of abstraction.
\end{itemize}
